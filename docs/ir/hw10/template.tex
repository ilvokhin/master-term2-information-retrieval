\documentclass[12pt]{article}

\usepackage{fullpage}
\usepackage{multicol, multirow}
\usepackage{tabularx}
\usepackage{standalone}
\usepackage{listings}
\usepackage{ulem}
\usepackage{amsmath}
\usepackage{pdfpages}
\usepackage[utf8]{inputenc}
\usepackage[russian]{babel}

\newcommand{\StudentName}{Ильвохин Дмитрий}
\newcommand{\Group}{1O-106М}
\newcommand{\CourseName}{Информационный поиск}
\newcommand{\LabNum}{10}
\newcommand{\Subject}{Поиск с метасимволами}
\newcommand{\PrepName}{Калинин А.\,Л.}

\begin{document}

\lstset
{
        language=Python,
        basicstyle=\footnotesize,% basic font setting
        extendedchars=\true
}

\begin{flushright}
\Large{
	\CourseName \\
	Лабораторная работа №\,\LabNum \\
	<<\Subject>> \\
  \StudentName, \Group \\
  Дата: \line(1,0){150} \\
  Подпись: \line(1,0){150} \\
}
\end{flushright}

\subsection*{Задание}
Реализовать поиск для запросов с метасимволами.

\subsection*{Исходный код}
\lstinputlisting[caption={Построение бора}]{../../../make_trie.py}

\lstinputlisting[caption={Поиск с метасимволами}]{../../../metasymbol_search.py}

\subsection*{Результаты}
\lstinputlisting[language=Bash]{metasymbol_search.log.txt}

\subsection*{Выводы}
Перебор всех суффиксов от узла --- не самая эффективная операция.
Возможно, для большего количества данных имело бы смысл предподcчитать
достижимые узлы для заданного.

\end{document}

